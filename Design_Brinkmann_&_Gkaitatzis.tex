\documentclass[10pt,a4paper]{report}
\usepackage[latin1]{inputenc}
\usepackage[ngerman]{babel}

\author{Brinkmann, Maximilian, 2319886, maximilian.brinkmann@haw-hamburg.de \\
Gkaitatzis, Andreas, 2320008, Gkaitatzis.A@web.de}

\title{Design Dokument \\PR1 Praktikum WS2016}
\date{ 	\begin{tabular}{|c|c|c|}
		\hline 
		Version & Datum & Anmerkung \\ 
		\hline 
		1.0 & 21.10.2016 & Erste Abgabe\\ 
		\hline 
		\end{tabular} }

\begin{document}
	\maketitle
	
	\section*{Aufgabe 1: Hallo Person 1 und Person 2}
	\subsection*{Design in Pseudocode}
	\subsubsection{Aktivit�ten:}
	- Ausgabe: Hallo Max und Andreas!
	
	\section*{Aufgabe 2: Fakult�t von n}
	\subsection*{Design in Pseudocode}
	\subsubsection*{Daten:}
	1.	ergebnis\\
	2.	n (Eingabe)\\
	3.	zaehler
	\subsubsection*{Aktivit�ten:}
	- Einlesen von n\\
	- ergebnis und z�hler gleich 1\\
	- Solange z�hler kleinergleich n, mach\\
	$\rightarrow$ ergebnis * z�hler\\
	$\rightarrow$ z�hler +1\\
	-Ausgabe: ergebnis
	
	\section*{Aufgabe 3: f(x) = x$^2$ + y$^2$}
	\subsection*{Design in Pseudocode}
	\subsubsection*{Daten:}
	1.	ergebnis\\
	2.	x (Eingabe)\\
	3.	y (Eingabe)
	\subsubsection*{Aktivit�ten:}
	- Einlessen von x\\
	- Einlesen von y\\
	- ergebnis = x*x + y*y\\
	- ergebnis Ausgeben
	
	\newpage
	
	\section*{Aufgabe 4: Satellitenzeit}
	\subsection*{Desgin in Pseudocode}
	\subsubsection*{Daten:}
	1.	anzahlSekunden (Eingabe)\\
	2.	restSekunden\\
	3.	tage\\
	4.	stunden\\
	5.	minuten
	\subsubsection*{Aktivit�ten:}
	- Einlesen von anzahlSekunden\\
	- tage = anzahlSekunden / 86400\\
		$\rightarrow$ Ein Tag hat 86400 Sekunden\\
	- restSekunden = anzahlSekunden \% 86400\\
		$\rightarrow$ Alle m�glichen Tage aus anzahlSekunden rausrechnen\\
	- stunden = restSekunden / 3600\\
		$\rightarrow$ Eine Stunde hat 300 Sekunden\\
	- restSekunden = anzahlSekunden \% 3600\
		$\rightarrow$ Alle m�gl. Stunden aus anzahlSekunden rausrechnen\\
	-minuten = restSekunden / 60\\
		$\rightarrow$ Eine Minute hat 60 Sekunden\\\
	- restSekunden = restSekunden \% 60\\
		$\rightarrow$ Alle Minuten aus restSekunden rausrechnen, �brig bleiben nur Sekunden, ohne Min, Stunden oder Tage\\
	-Ausgabe tage, stunden, minuten, restSekunden
				
	
	
	
	
	
	
	
	
	
\end{document}